\documentclass[11pt]{article}

\usepackage[T1]{fontenc}
\usepackage[latin9]{luainputenc}
\usepackage[letterpaper]{geometry}
\geometry{verbose}
\usepackage{amsfonts}
\usepackage{babel}
\usepackage{ulem}

\usepackage{extarrows}
\usepackage[colorlinks]{hyperref}
\usepackage{listings}
\usepackage{xcolor}
\usepackage[ruled,linesnumbered]{algorithm2e}
\usepackage{float}
\usepackage{amsmath,graphicx}
\usepackage{subfigure} 
\usepackage{cite}
\usepackage{amsthm,amssymb,amsfonts}
\usepackage{textcomp}
\usepackage{bm,pifont}
\usepackage{booktabs}
\usepackage{listings}
\usepackage{xparse}
\usepackage{cleveref}
\usepackage{xcolor}
\usepackage[center]{caption}


\begin{document}

\begin{center}
	\textbf{\LARGE{Numerical analysis(SI211)$_\text{Fall 2021-22}\,$ Homework 3 } } \\
	\texttt{Prof. Jiahua Jiang }\\
	\texttt{\textbf{Name:} xxx }  \quad 
	\texttt{\textbf{Student No.:} xxx }  \quad 
	\texttt{\textbf{E-mail:}} \texttt{ xxx@shanghaitech.edu.cn}
\par\end{center}

\noindent
\rule{\linewidth}{0.4pt}
{\bf {\large Acknowledgements:}}
\begin{enumerate} 
    \item Deadline: \textcolor{red}{\textbf{2021-12-24 11:59:00}}, no late submission is allowed.
    \item No handwritten homework is accepted. You should submit your homework in \textcolor{blue}{Blackboard} with \textcolor{blue}{PDF} format, we recommend you use \LaTeX. 
    \item Giving your solution in English, solution in Chinese is not allowed.
    \item Make sure that your codes can run and are consistent with your solutions, you can use any  programming language. 
    \item Your PDF should be named as "{\sf your\_student\_id+HW3.pdf}", package all your codes into "{\sf your\_student\_id+Code3.zip}" and upload. \textcolor{blue}{ Don't put your PDF in your code file} 
    \item \color{blue}{All the results from your code should be shown in pdf but please do not inset your code into \LaTeX }.
    \item Plagiarism is not allowed. Those plagiarized solutions and codes will get 0 point. \textcolor{red}{
    If the results on the pdf are inconsistent with the results of code, your coding problem will get 0 point.}
    \end{enumerate}
\rule{\linewidth}{0.4pt}

\newpage
\begin{enumerate}
% % ____________________________________ pro_1 ____________________________________
  \item \textbf{Euler's Method}(\textcolor{blue}{20 points.})\\ 
  For initial-value problem:
  \begin{equation}
  \begin{aligned}
      &y'(x)=ax+b\\
      &y(0) = 0,
  \end{aligned}
 \end{equation}
 use Euler's method and Taylor's method of order $2$ to derive the approximation of $y_{i+1}$ with step size $h$ respectively. Besides, compare your results with the exact solution $y=\frac{1}{2}ax^2+bx$ (i.e. compare $y_{i+1}$ and $y(x_{i+1})$). \\
\textbf{Solution:}\\

    
% % ____________________________________ pro_2 ____________________________________
\newpage
\item \textbf{Runge-Kutta Methods}(\textcolor{blue}{20 points.})\\
Prove the following Runge-kutta method is of order 3(i.e. has truncation error $\mathcal{O}(h^4)$)
\begin{equation}
    \begin{aligned}
        y_{i+1} &= y_i+\frac{h}{4}(K_1+3K_3)\\
        K_1 &= f(x_i,y_i)\\
        K_2 &= f(x_i+\frac{h}{3},y_i+\frac{h}{3}K_1)\\
        K_3 &= f(x_i+\frac{2}{3}h,y_i+\frac{2}{3}hK_2)
    \end{aligned}
\end{equation}
\newline \textbf{Solution:}\\

% % ____________________________________ pro_3 ____________________________________
\newpage


\item \textbf{\textcolor{red}{Coding}Runge-Kutta Order Four}(\textcolor{blue}{20 points.}) 
Use Runge-Kutta Fourth-Order method to solve the following initial-value problem:
\begin{equation}
\begin{aligned}
    y'(x)&=x+y(0\leq x\leq1)\\
    y(0)&=1.
\end{aligned}
\end{equation}
The exact solution of the problem is $y(x) = -x-1+2e^x$.
With step size $h=0.1$, give your predictions within the interval $x\in [0,1]$. List the Runge-Kutta 4 method results and their errors in the following table. 
\begin{center}
\begin{tabular}{||c c c c||} 
 \hline
 $x_i$ & Exact($y_i = y(x_i)$) & Runge-Kutta Order 4 ($w_i$) & Error($|y_i-w_i|$) \\ [0.5ex] 
 \hline\hline
 0.0 & 1.0 & 1.0 & 0 \\
 \hline
 ... & ... & ... & ... \\
 \hline
  1.0 & ... & ... & ... \\[1ex] 
 \hline
\end{tabular}
\end{center}



\end{enumerate}
\end{document}