\documentclass[11pt]{article}

\usepackage[T1]{fontenc}
\usepackage[latin9]{luainputenc}
\usepackage[letterpaper]{geometry}
\geometry{verbose}
\usepackage{amsfonts}
\usepackage{babel}
\usepackage{ulem}

\usepackage{extarrows}
\usepackage[colorlinks]{hyperref}
\usepackage{listings}
\usepackage{xcolor}
\usepackage[ruled,linesnumbered]{algorithm2e}
\usepackage{float}
\usepackage{amsmath,graphicx}
\usepackage{subfigure} 
\usepackage{cite}
\usepackage{amsthm,amssymb,amsfonts}
\usepackage{textcomp}
\usepackage{bm,pifont}
\usepackage{booktabs}
\usepackage{listings}
\usepackage{xparse}
\usepackage{cleveref}
\usepackage{xcolor}
\usepackage[center]{caption}


\begin{document}

\begin{center}
	\textbf{\LARGE{Numerical analysis(SI211)$_\text{Fall 2021-22}\,$ Homework 2 } } \\
	\texttt{Prof. Jiahua Jiang }\\
	\texttt{\textbf{Name:} xxx }  \quad 
	\texttt{\textbf{Student No.:} xxx }  \quad 
	\texttt{\textbf{E-mail:}} \texttt{ xxx@shanghaitech.edu.cn}
\par\end{center}

\noindent
\rule{\linewidth}{0.4pt}
{\bf {\large Acknowledgements:}}
\begin{enumerate} 
    \item Deadline: \textcolor{red}{\textbf{2021-12-10 11:59:00}}, no late submission is allowed.
    \item No handwritten homework is accepted. You should submit your homework in \textcolor{blue}{Blackboard} with \textcolor{blue}{PDF} format, we recommend you use \LaTeX. 
    \item Giving your solution in English, solution in Chinese is not allowed.
    \item Make sure that your codes can run and are consistent with your solutions, you can use any  programming language. 
    \item Your PDF should be named as "{\sf your\_student\_id+HW2.pdf}", package all your codes into "{\sf your\_student\_id+\_Code2.zip}" and upload. \textcolor{blue}{ Don't put your PDF in your code file} 
    \item \color{blue}{All the results from your code should be shown in pdf but please do not inset your code into \LaTeX }.
    \item Plagiarism is not allowed. Those plagiarized solutions and codes will get 0 point. \textcolor{red}{
    If the results on the pdf are inconsistent with the results of code, your coding problem will get 0 point.}
    \end{enumerate}
\rule{\linewidth}{0.4pt}

\newpage
\begin{enumerate}
% % ____________________________________ pro_1 ____________________________________
  \item \textbf{Numerical differentiation}(\textcolor{blue}{10 points.}) 
  Assume $f(x)\in C^3$, there are 3 points $f(x_0-\alpha h),f(x_0),f(x_0+ h)$ with $\alpha>0$. 
  \begin{enumerate}
      \item use Lagrange Polynomials to construct an approximation for $f''(x_0)$,
      \item evaluate the approximation error and find the approximation order.
  \end{enumerate}   
\textbf{Solution:}
    
% % ____________________________________ pro_2 ____________________________________
\item \textbf{Richardson extrapolation}(\textcolor{blue}{10 points.}) The $f'(x_0)$ can be expressed as 
\begin{equation}
\label{eq:3order_forward_diff}
    f'(x_0) = \frac{1}{h}(f(x_0+h) - f(x_0)) - \frac{h}{2}f''(x_0) -\frac{h^2}{6}f'''(x_0) + O(h^3).
\end{equation} 
Use idea of Richardson extrapolation to derive a 3 point formula for $f'(x_0)$ with $O(h^2)$ error. 
(Hint:Replace step size $h$ with $2h$.)
\newline \textbf{Solution:}

% % ____________________________________ pro_3 ____________________________________
\item \textbf{Elements of Numerical Integration}(\textcolor{blue}{20 points.})
For the integral $\int_2^6 \frac{1}{1+x}\mathrm{d}x$, use numerical integration methods to approximate.
\begin{enumerate}
    \item Given only the values of $f(x)$ at $x=$ 2,3,4, 5 and 6, use the Midpoint Rule, the Trapezoidal rule to approximate with the smallest step size possible.
    \item Use Romberg integration to compute $R_{3,3}$.
    \item Use Gaussian quadrature with $n=2$ to approximate the integral.
\end{enumerate}
\textbf{Solutions:}

% % ____________________________________ pro_4 ____________________________________
\item \textbf{\textcolor{red}{Coding} of Simpson's Rule}(\textcolor{blue}{20 points.}) For integral
\begin{equation}
    \int\limits_{0}^{4} e^x \mathrm{d}x,
\end{equation}
we write it in the form:
\begin{equation}
    \int\limits_{0}^{4} e^x \mathrm{d}x = \sum_{i=0}^{N-1}\{\int_{4i/N}^{4(i+1)/N}e^x\mathrm{d}x\},
\end{equation}
 then apply Simpson's rule on each part separately and sum up the results. You need to:
 \begin{itemize}
     \item Plot the actual error of this integral approximation versus $N$ for $N \in\{1,2\ldots,100\}$.
     \item Derive a theoretical bound on the integral approximation in dependence on $N$ and plot this upper bound, too.
 \end{itemize}
\textbf{Solutions:}

% % ____________________________________ pro_Bonus ____________________________________
\item \textbf{\textcolor{red}{Bonus Coding}(Multiple Integrals) }(\textcolor{blue}{20 points.}) For the 
  \begin{equation}\label{muti-int}
    \iint\limits_{\mathcal{D}} e^{-xy}\mathrm{d}x\mathrm{d}y
  \end{equation}
  with $\mathcal{D}=:\{0\le x \le 1.5, 0\le y \le 2 \} $.
 \begin{itemize}
     \item Use Composite Simpson's rule with $n=6 $ and $m=8 $, {i.e.}, $h_{x} = \frac{1.5}{6},h_{y} = \frac{2}{8} $ to approximate \eqref{muti-int}.
     \textcolor{blue}{(\textbf{Note:} your code should input the box range $\mathcal{D} $ and the integers $n,m $, you can use the Example.1 from the page-$239 $ to debug)}.  
    %  \item Use Gaussian quadrature with $n=3$ in both dimensions to approximate \eqref{muti-int}.
\end{itemize}
\textbf{Solution:}
 


\end{enumerate}
\end{document}