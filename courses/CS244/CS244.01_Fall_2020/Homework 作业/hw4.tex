\documentclass{article}

\usepackage{fancyhdr}
\usepackage{extramarks}
\usepackage{amsmath}
\usepackage{amsthm}
\usepackage{tikz}
\usepackage{enumerate}
\usepackage{amssymb}
\usetikzlibrary{automata, positioning}
\usepackage{mdwtab}
\usepackage{syntax}
\topmargin=-0.45in
\evensidemargin=0in
\oddsidemargin=0in
\textwidth=6.5in
\textheight=9.0in
\headsep=0.25in
\setlength{\grammarparsep}{0.05cm}   % vertical distance between production rules
\setlength{\grammarindent}{0.01cm}      % horizontal indent distance
\linespread{1.1}

\pagestyle{fancy}
\lhead{\hmwkAuthorName\ -\ \hmwkAuthorID}
\chead{\hmwkClass: Homework \hmwkNo}
\rhead{\firstxmark}
\lfoot{\lastxmark}
\cfoot{\thepage}

\renewcommand\headrulewidth{0.4pt}
\renewcommand\footrulewidth{0.4pt}

\newcommand{\enterProblemHeader}[1]{
    \nobreak\extramarks{}{Problem \arabic{#1} continued on next page\ldots}\nobreak{}
    \nobreak\extramarks{Problem \arabic{#1} (continued)}{Problem \arabic{#1} continued on next page\ldots}\nobreak{}
}

\newcommand{\exitProblemHeader}[1]{
    \nobreak\extramarks{Problem \arabic{#1} (continued)}{Problem \arabic{#1} continued on next page\ldots}\nobreak{}
    \stepcounter{#1}
    \nobreak\extramarks{Problem \arabic{#1}}{}\nobreak{}
}

\setcounter{secnumdepth}{0}
\newcounter{homeworkProblemCounter}
\setcounter{homeworkProblemCounter}{1}
\nobreak\extramarks{Problem \arabic{homeworkProblemCounter}}{}\nobreak{}

\newenvironment{homeworkProblem}[1][-1]{
    \ifnum#1>0
        \setcounter{homeworkProblemCounter}{#1}
    \fi
    \section{Problem \arabic{homeworkProblemCounter}}
    \enterProblemHeader{homeworkProblemCounter}
}{
    \exitProblemHeader{homeworkProblemCounter}
}

\newenvironment{solution}{
    \subsection{Solution}
}

\newcommand{\hmwkNo}{4}
\newcommand{\hmwkDueDate}{November 12, 2020 at 11:59 p.m.}
\newcommand{\hmwkClass}{CS244 Theory of Computation}
\newcommand{\hmwkClassInstructor}{Fu Song}
\newcommand{\hmwkAuthorName}{Name}
\newcommand{\hmwkAuthorID}{ID}
\newcommand{\inter}{\mathord{} \cap \mathord{}}
\newcommand{\union}{\mathord{} \cup \mathord{}}
\renewcommand{\SS}{\ensuremath{\Sigma}}
\newcommand{\SSS}{\ensuremath{\Sigma\kstar}}
\newcommand{\SSE}{\ensuremath{\Sigma_{\varepsilon}}}
\newcommand{\st}[1]{\mbox{\texttt{#1}}}
\newcommand{\GG}{\ensuremath{\Gamma}}
\newcommand{\GGE}{\ensuremath{\Gamma_{\hspace{-.15em}\varepsilon}}}
\newcommand{\tTk}[1]{\langle \mbox{\textsc{#1}} \rangle}
\newcommand{\cfgor}{\ | \ }
\newcommand{\cfgarrow}{\rightarrow}
\newcommand{\lcfgarrow}{\rightarrow}
\newcommand{\set}[1]{\{#1|\ }  %}
\newcommand{\setb}[1]{\{\brk{#1}| \ }
\newcommand{\setend}{\ensuremath{\}}}
\newcommand{\brk}[1]{\langle #1 \rangle}
\newcommand{\lang}[1]{\ensuremath{\text{\textit{#1}}}}
\newcommand{\defin}[1]{\textit{\textbf{{\boldmath #1}}}}
\newcommand{\NP}{\mathrm{NP}}
\renewcommand{\P}{\mathrm{P}}
\title{
    \vspace{-0.4in}
    \textmd{\textbf{\hmwkClass \\ Homework \hmwkNo}}\\
    \normalsize\vspace{0.1in}\small{Due: \hmwkDueDate}\\
}

\author{\hmwkAuthorName\ -\ \hmwkAuthorID}
\date{}

\begin{document}

\maketitle
\thispagestyle{fancy}

You may discuss this assignment with other students and work
on the problems together. However, your write-up should be your own individual work and you should indicate in your submission who you worked with, if applicable. You should use the {\LaTeX} template provided by us to write your solution and submit the generated PDF file into Gradescope. \\

I worked with: (Name, ID), (Name, ID), \ldots \\

Let $\Sigma = \{\mathsf{0}, \mathsf{1}\}$ if not otherwise specified.

\begin{homeworkProblem}
	Say that a variable $A$ in $\mathsf{CFG}$ $G$ is \defin{redundant}
	if removing it and its associated rules leaves $L(G)$ unchanged.
	Let $\lang{REDUNDANT}_{\mathsf{CFG}}=\setb{G,A}A$ is a redundant variable
	in $G$\setend.  
	Show that $\lang{REDUNDANT}_{\mathsf{CFG}}$ is \textbf{undecidable}.
\end{homeworkProblem}

\begin{homeworkProblem}
	Consider the problem of testing whether a $\mathsf{TM}$ accepts the empty string $\epsilon$. Formally, let $\lang{EPSILON}_{\mathsf{TM}}=\{\langle T \rangle \mid T \text{ is a }\mathsf{TM} \text{ that accept } \epsilon\}$.
	\begin{enumerate}[(a)]
		\item Use a \textbf{reduction} or the \textbf{recursion theorem} to prove that $\lang{EPSILON}_\mathsf{TM}$ is \textbf{undecidable}.
		\item Answer YES or NO and give a brief reason for your answer. \\
				 Is $A_\mathsf{TM}$ \textbf{mapping reducible} to $\overline{\lang{EPSILON}_\mathsf{TM}}$?
	\end{enumerate}
\end{homeworkProblem}


\begin{homeworkProblem}
	Show that $\lang{EQ}_{\mathsf{TM}}\nleq_m\overline{\lang{EQ}_{\mathsf{TM}}}$.
\end{homeworkProblem}

\begin{homeworkProblem}
	Let $\textit{SET-SPLITTING} = \{ \langle S, C \rangle \mid S$ is a finite set and $C = \{ C_1, \dotsc, C_k \}$ is a collection of subsets of $S$, where the elements of $S$ can be colored red or blue so every $C_i$ has at least one red element and at least one blue element $\}$. Show that $\textit{SET-SPLITTING}$ is \textbf{NP-complete}.
\end{homeworkProblem}

\end{document}